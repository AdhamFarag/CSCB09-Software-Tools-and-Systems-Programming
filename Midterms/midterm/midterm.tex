\long\def\ignore#1{} % to comment out parts of the file

%% New conditional to allow formatting of either exam paper or solutions.
\newif\ifsolutions % starts out false by default
%\solutionstrue % comment out to format solutions and marking scheme

%% New conditional to allow formatting of marking sheet
\newif\ifmarking % starts out false by default
%\markingtrue % comment out to format marking scheme

\documentclass[11pt]{article}
\usepackage{fancyhdr,rotating,multicol,graphicx, verbatim}
\usepackage{amsmath,amssymb}
%\usepackage{amssymb}

%% Dirty trick so that math symbols will be bold when text is.
\let\seiresfb=\bfseries
\def\bfseries{\seiresfb\boldmath}

%% Custom page layout.
\setlength{\textheight}{\paperheight}
\addtolength{\textheight}{-2in}
\setlength{\topmargin}{-.5in}
\addtolength{\topmargin}{-.25\headheight}
\setlength{\headsep}{.5in}
\addtolength{\headsep}{-.75\headheight}
\setlength{\footskip}{.5in}
\setlength{\textwidth}{\paperwidth}
\addtolength{\textwidth}{-1.5in}
\setlength{\oddsidemargin}{-.25in}
\setlength{\evensidemargin}{-.25in}
\setlength{\parskip}{1ex}
\setlength{\parindent}{0pt}
\flushbottom

\allowdisplaybreaks

%% Spacing macros (with default argument).
\newcommand{\vstretch}[1][1]{\vspace*{\stretch{#1}}}
\newcommand{\hstretch}[1][1]{\hspace*{\stretch{#1}}}

%% Macros to generate question and part numbers automatically.
\newcounter{temp}
\newcounter{bonusmarks}
\setcounter{bonusmarks}{1}
\newcounter{totalmarks}
\setcounter{totalmarks}{0}
\newcounter{questionnumber}
\setcounter{questionnumber}{0}
\newcounter{subquestionnumber}[questionnumber]
\setcounter{subquestionnumber}{0}
\renewcommand{\thesubquestionnumber}{\alph{subquestionnumber}}
\makeatletter
\newcommand{\bonus}[2][]%
  {\ifx\empty#2\empty\else\addtocounter{bonusmarks}{#2}\fi
   \@startsection{bonus}{6}{0pt}{-1.25ex plus -2.5ex minus -.75ex}
   {.75ex plus .75ex minus .5ex}{\normalfont}*
     {{\bfseries\Large Bonus. } #1 {\scshape
      \ifx\empty#2\empty(continued)\else[#2 mark\ifnum#2>1s\fi]\fi}}}
\newcommand{\question}[2][]%
  {\ifx\empty#2\empty\else\addtocounter{totalmarks}{#2}
   \setcounter{temp}{#2}\addtocounter{temp}{-1}\refstepcounter{temp}
   \addtocounter{questionnumber}{1}\label{Q\thequestionnumber Marks}
   \addtocounter{questionnumber}{-1}\refstepcounter{questionnumber}\fi
   \@startsection{question}{7}{0pt}{-1.25ex plus -2.5ex minus -.75ex}
   {.75ex plus .75ex minus .5ex}{\normalfont}*
     {{\bfseries\Large Question \thequestionnumber. } #1 {\scshape
      \ifx\empty#2\empty(continued)\else[#2 mark\ifnum#2>1s\fi]\fi}}}
\newcommand{\subquestion}[2][]%
  {\ifx\empty#2\empty\else\refstepcounter{subquestionnumber}\fi
   \@startsection{subquestion}{8}{0pt}{-.75ex plus -1.5ex minus -.5ex}
   {.5ex plus .5ex minus .25ex}{\normalfont}*
     {{\bfseries\large Part (\thesubquestionnumber) } #1 {\scshape
      \ifx\empty#2\empty(continued)\else[#2 mark\ifnum#2>1s\fi]\fi}}}
\newcommand{\subquestionnospace}[2][]%
  {\ifx\empty#2\empty\else\refstepcounter{subquestionnumber}\fi
   \@startsection{subquestion}{8}{0pt}{-.75ex plus -1.5ex minus -.5ex}
   {-.5ex plus -.5ex minus -.25ex}{\normalfont}*
     {{\bfseries\large Part (\thesubquestionnumber) } #1 {\scshape
      \ifx\empty#2\empty(continued)\else[#2 mark\ifnum#2>1s\fi]\fi}}}
\newcommand\heading%
  {\@startsection{heading}{9}{0pt}{-.75ex plus -1.5ex minus -.5ex}
   {-.5em plus -.5em minus -.25em}{\normalfont\scshape}*}
\makeatother

%% Macro to produce a list of question numbers and marks.
\newcommand{\markingguide}[1][]%
  {\setcounter{temp}{0\ref{EndOfExam}}%
   \begin{tabular}{r@{\ \protect\urule{.5in}/}r}
%  \multicolumn{2}{c}{\textsc{\large Marking Guide}}\\[4ex]
   \ifnum0<0\ref{BonusMarks}
      \begin{tabular}[b]{@{}l@{}}
         \textsc{Bonus}\\   \textsc{Marks}
      \end{tabular}: & \ref{BonusMarks}\\[4ex]
   \fi\ifx\empty#1\empty
      \ifnum0<0\ref{Q0Marks}\recmarkingguide{0}
      \else\recmarkingguide{1}\fi
   \else\recmarkingguide{#1}\fi\\[4ex]
      \textsc{TOTAL}: & \ref{TotalMarks}
   \end{tabular}}
\newcommand{\recmarkingguide}[1]%
  {\ifnum\thetemp>#1\addtocounter{temp}{-1}
   \recmarkingguide{#1}\\[2ex]\addtocounter{temp}{1}\fi
   \#~\thetemp: & \ref{Q\thetemp Marks}}

%% Macros for drawing "underline" rules and half-boxes.
\newcommand{\urule}[2][.5pt]{\rule[-2pt]{#2}{#1}} % "underline" rule
\newcommand{\hb}{\urule{1.5em}}                   % horizontal bar
\newcommand{\vb}{\urule[1ex]{.5pt}}               % vertical bar
\newcommand{\studentnumberboxes}%
  {\vb\hb\vb\hb\vb\hb\vb\hb\vb\hb\vb\hb\vb\hb\vb\hb\vb\hb\vb}

%% Macros for making checkboxes and checklists.
\newcommand{\cbox}[1]{{\fboxrule.125ex\fboxsep0ex
   \framebox[2.75ex]{\rule[-.75ex]{0ex}{2.75ex}\smash{#1}}}}
\newcommand{\cmark}{\raisebox{.0625ex}{$\surd$\kern.125ex}}
\newcommand{\xmark}{\raisebox{-.1875ex}{\kern.0625ex{X}}}
\newcommand{\checkbox}{{\sf\cbox{}}}
\newcommand{\checkedbox}{{\sf\cbox{\cmark}}}
\newcommand{\markedbox}{{\sf\cbox{\xmark}}}
\newenvironment{checklist}{\begin{list}{\checkbox}{}}{\end{list}}
\newcommand{\checkeditem}{\item[\checkedbox]}
\newcommand{\markeditem}{\item[\markedbox]}

%% Itemize-like environment with "condensed" interline spacing.
\newenvironment{points}[1][$\bullet$]
  {\begin{list}{{#1}}
     {\setlength{\parsep}{0pt plus 0.5pt minus 0pt}
      \setlength{\topsep}{0pt plus 0.5pt minus 0pt}
      \setlength{\itemsep}{0pt plus 0.5pt minus 0pt}}}
  {\end{list}}

%% Various formats.
\let\longem=\textsl
\let\strong=\textbf
\let\code=\texttt

%% Various symbols.
\newcommand{\caret}{\symbol{"5E}} % caret ^
\newcommand{\lcurl}{\symbol{"7B}} % left curly bracket {
\newcommand{\rcurl}{\symbol{"7D}} % right curly bracket }
\newcommand{\ie}{\emph{i.e.}}
\newcommand{\eg}{\emph{e.g.}}
\newcommand{\bigOh}{\mathcal{O}}

%% Custom headers and footers.
\pagestyle{fancyplain}
\let\headrule=\empty
\let\footrule=\empty
\lhead{\fancyplain{}{{CSC\,209H5\,S}}}
\chead{\fancyplain{}
   {{\scshape\large Midterm\ifsolutions\ --- Solutions\fi}}}
\rhead{\fancyplain{}{{Winter 2015}}}
\lfoot{\fancyplain{{Total Pages = \pageref{EndOfExam}}}
   {{\small Student \#: }{\tiny\studentnumberboxes}}}
\cfoot{\fancyplain{{Page \thepage}}
   {{Page \thepage\ of \pageref{EndOfExam}}}}
\rfoot{{\scshape\small cont'd$\ldots$}}


\begin{document}

%% Cover Page %%%%%%%%%%%%%%%%%%%%%%%%%%%%%%%%%%%%%%%%%%%%%%%%%%%%%%%%%%%%%%
\ifsolutions
\else
\thispagestyle{plain}

\vspace*{-.2in}
\parbox{2.75in}{
\centering\large
 {CSC\,209H5\,S} {2015} {Midterm}\\
 Duration --- {50 minutes}\\
      Aids allowed: none
}%parbox
\hfill
  \parbox{4in}{ \begin{tabular}{@{}rl@{}}
   \strong{Student Number:} & \studentnumberboxes \\[2ex]
   % \strong{Lab day, time, room:} & \hrulefill
\end{tabular}
}%parbox
%\end{center}

\begin{center}
\begin{tabular}{@{}rl@{\qquad}rl@{}}
       \strong{Last Name:} & \urule{2in}
       \hspace{.2in}\strong{First Name:} & \urule{2in} \\
\end{tabular}
\end{center}

\begin{center}
\begin{tabular}{cc}
    \strong{Lecture Section:}  1 &
    \hspace{1in}\strong{Instructor: Daniel Zingaro}  %\urule{2in} &
\end{tabular}
\end{center}

  \rule[.1875\baselineskip]{\textwidth}{1pt}
\begin{center}
  \vspace*{-.25in} \Large \longem{Do \strong{not} turn this page until you
    have received the signal to start.} \\
  (Please fill out the identification section above, \textbf{write your name on the back of the test}, and read the instructions below.) \\
 {\slshape\Large Good Luck!}
  \rule[.1875\baselineskip]{\textwidth}{1pt}\\
\end{center}

\normalsize
\begin{center}
\vspace*{-.2in}
  \parbox{5in}{ This midterm consists of \ref{EndOfExam}
    {question\ifnum1<0\ref{EndOfExam}s\fi} on \pageref{EndOfExam}
    {page\ifnum1<0\pageref{EndOfExam}s\fi} (including this one).
    \longem{When you receive the signal to start, please make sure that
      your copy is complete.}
    
    %    For 1 bonus mark write your student number at the bottom of pages
%    2-\pageref{EndOfExam} of this test and provide your lab day, time, and room.
   
    If you use any space for rough work, indicate clearly
    what you want marked.

}%parbox
\qquad
\markingguide\\
  \rule[.1875\baselineskip]{\textwidth}{1pt}
\end{center}

\fi%solutions
%%%%%%%%%%%%%%%%%%%%%%%%%%%%%%%%%%%%%%%%%%%%%%%%%%%%%%%%%%%%%% Cover Page %%

% Question 1
\newpage 
\question{6}

For each code snippet, do one of two things:
\begin{itemize}
\item If the code runs and is well-defined, give its output.
\item Otherwise, state the problem, and explain why the code is not well-defined or does not run.
\end{itemize}

\subquestionnospace{2}

\begin{small}
\verbatiminput{bug1.c}
\end{small}

\vfill

\subquestionnospace{2}

\begin{small}
\verbatiminput{bug2.c}
\end{small}

\vfill

\newpage

\subquestionnospace{2}

\begin{small}
\verbatiminput{bug3.c}
\end{small}

\vfill

\newpage
\question{5}

\subquestionnospace{3}

I type this command at the shell: \\
\verb|sort <junk > result| \\
Explain what the shell has to do to execute this command. Refer to specific system calls in your answer.

\vfill

\subquestionnospace{2}

Write a \verb|makefile|, to be processed by \verb|make|, that prints the word \verb|hello| iff the file \verb|dan| does not exist.

\vfill

\newpage
\question{6}

Here is the \verb|struct block| from Assignment 2.

\begin{small}
\begin{verbatim}
struct block {
  void *addr; /*start address of memory for this block */
  int size;
  struct block *next;
};
\end{verbatim}
\end{small}

Suppose \verb|freelist| is already initialized and contains zero or more blocks, and that \verb|freelist| is sorted by increasing \verb|addr| values.

One thing we might like to do is {\bf coalesce} blocks of free memory into larger blocks, so that future calls to \verb|smalloc| are more likely to succeed.

For example, suppose that the first block in \verb|freelist| has \verb|addr = 500| and \verb|size = 100|. Suppose also that the second block in \verb|freelist| has \verb|addr = 600| and \verb|size = 50|. What we can do is replace this by a single block with \verb|addr = 500| and \verb|size = 150|.

Write function \verb|coalesce| below that coalesces all possible blocks in \verb|freelist|. For full marks, you must not have any memory leaks.

\begin{small}
\begin{verbatim}
void coalesce (block *freelist) {
\end{verbatim}
\end{small}

\newpage

[more space for your \verb|coalesce| function]

\newpage

\newpage
\question{3}

Write the following function that copies all bytes from \verb|infd| to \verb|outfd|.

\begin{small}
\begin{verbatim}
int copy_fd(int infd, int outfd) {
  //infd and outfd are properly-opened FDs.
  //Copy all bytes from infd to outfd.
  //Return number of bytes copied.
\end{verbatim}
\end{small}


\newpage
{\huge \bf Function Prototypes}

\begin{small}
\begin{verbatim}
int printf(const char *format, ...);

size_t strlen(const char *s);
char *strcpy(char *dest, const char *src);
char *strncpy(char *dest, const char *src, size_t n);

ssize_t read(int fd, void *buffer, size_t maxbytes);
ssize_t write(int fd, void *buffer, size_t maxbytes);

size_t fread(void *ptr, size_t elem_size, 
             size_t num_elem, FILE *stream);

size_t fwrite(const void *ptr, size_t elem_size,
              size_t num_elem, FILE *stream);
\end{verbatim}
\end{small}

\ifsolutions
\else
\newpage
\newpage
\textit{[Use the space below for rough work. This page will not be marked unless you clearly indicate the part of your work that you want us to mark.]}
%\newpage
%\textit{[Use the space below for rough work. This page will not be marked unless you clearly indicate the part of your work that you want us to mark.]}


\fi

\ifsolutions
\else
\newpage
\begin{center}
\begin{tabular}{@{}rl@{\qquad}rl@{}}
       \strong{Last Name:} & \urule{2in}
       \hspace{.2in}\strong{First Name:} & \urule{2in} \\
\end{tabular} \\
\end{center}
\vspace*{1cm}
%{\huge \bf Function Prototypes}

\begin{small}
\begin{verbatim}
int printf(const char *format, ...);

size_t strlen(const char *s);
char *strcpy(char *dest, const char *src);
char *strncpy(char *dest, const char *src, size_t n);

ssize_t read(int fd, void *buffer, size_t maxbytes);
ssize_t write(int fd, void *buffer, size_t maxbytes);

size_t fread(void *ptr, size_t elem_size, 
             size_t num_elem, FILE *stream);

size_t fwrite(const void *ptr, size_t elem_size,
              size_t num_elem, FILE *stream);
\end{verbatim}
\end{small}

%\textbf{Use the remainder of this page for rough work.}
\fi

\lfoot{\fancyplain{{Total Pages = \pageref{EndOfExam}}}}

\ifsolutions
\else
%\center{\large\textbf{You can tear this page off, but we will collect it.
%    You must fill in your student number if you tear it off, and you will
%    lose 20\% if you keep this page.}}
\fi

\vstretch

%\newpage
%{\huge \bf Function Prototypes}

\begin{small}
\begin{verbatim}
int printf(const char *format, ...);

size_t strlen(const char *s);
char *strcpy(char *dest, const char *src);
char *strncpy(char *dest, const char *src, size_t n);

ssize_t read(int fd, void *buffer, size_t maxbytes);
ssize_t write(int fd, void *buffer, size_t maxbytes);

size_t fread(void *ptr, size_t elem_size, 
             size_t num_elem, FILE *stream);

size_t fwrite(const void *ptr, size_t elem_size,
              size_t num_elem, FILE *stream);
\end{verbatim}
\end{small}


%% Last Page %%%%%%%%%%%%%%%%%%%%%%%%%%%%%%%%%%%%%%%%%%%%%%%%%%%%%%%%%%%%%%%
%% Print total marks at the bottom of the last page.
%\vfil\vskip.5in\vskip-\baselineskip
%\centerline{Total Marks = \thetotalmarks}
%% Labels for marks, last page, and last question number.
\addtocounter{totalmarks}{-1}\refstepcounter{totalmarks}\label{TotalMarks}%
%\addtocounter{bonusmarks}{-1}\refstepcounter{bonusmarks}\label{BonusMarks}%
\addtocounter{questionnumber}{-1}\refstepcounter{questionnumber}%
\label{EndOfExam}%
%% Foot for last page of test paper.
\rfoot{{\scshape\small End of \ifsolutions Solutions\else Examination\fi}}%

\end{document}
