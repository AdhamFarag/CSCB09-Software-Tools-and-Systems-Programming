\question{6}

Here is the \verb|struct block| from Assignment 2.

\begin{small}
\begin{verbatim}
struct block {
  void *addr; /*start address of memory for this block */
  int size;
  struct block *next;
};
\end{verbatim}
\end{small}

Suppose \verb|freelist| is already initialized and contains zero or more blocks, and that \verb|freelist| is sorted by increasing \verb|addr| values.

One thing we might like to do is {\bf coalesce} blocks of free memory into larger blocks, so that future calls to \verb|smalloc| are more likely to succeed.

For example, suppose that the first block in \verb|freelist| has \verb|addr = 500| and \verb|size = 100|. Suppose also that the second block in \verb|freelist| has \verb|addr = 600| and \verb|size = 50|. What we can do is replace this by a single block with \verb|addr = 500| and \verb|size = 150|.

Write function \verb|coalesce| below that coalesces all possible blocks in \verb|freelist|. For full marks, you must not have any memory leaks.

\begin{small}
\begin{verbatim}
void coalesce (block *freelist) {
\end{verbatim}
\end{small}

\newpage

[more space for your \verb|coalesce| function]

\newpage
